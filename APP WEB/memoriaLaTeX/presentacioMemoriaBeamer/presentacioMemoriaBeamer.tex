\documentclass{beamer}

% Tema de la presentación
\usetheme{Berlin} % Puedes cambiarlo a: Warsaw, Berlin, AnnArbor, etc.
\usepackage{graphicx}
\usepackage{multicol}




% Información del título
\title[Creación de un dashboard para usuarios del ticket digital de Mercadona]{Creación de un dashboard para usuarios del ticket digital de Mercadona con visualización gráfica de datos: evolución de precios por producto, gastos por categoría de alimentación y ventanas temporales de gastos}
\author{Santiago Sánchez Sans}
\institute{IES Abastos}
\date{6 junio 2025} % o una fecha específica


\setbeamertemplate{footline}[frame number]

\begin{document}



	% Portada
	\begin{frame}
		\titlepage
	\end{frame}
	
	% Índice UNA COLUMNA
	\begin{frame}
		\frametitle{Contenido}
		\tableofcontents
	\end{frame}
	
	% Índice DOS COLUMNAS
	%\begin{frame}
	%	\frametitle{Contenido}
	%	\begin{multicols}{2}
		%		\tableofcontents
		%	\end{multicols}
	%\end{frame}

	
	
	% INDICE CON DIFUMINADOS A CADA SECCION
	\AtBeginSection[]{
		\begin{frame}
			\frametitle{Contenidos}
			\tableofcontents[currentsection]
		\end{frame}
	}
	
	% INDICE CON DIFUMINADOS A CADA SECCION (2 COLS)
	%\AtBeginSection[]{
		%	\begin{frame}
			%		\frametitle{Contenidos}
			%		\begin{multicols}{2}
				%			\tableofcontents[currentsection]
				%		\end{multicols}
			%	\end{frame}
		%}

	
	
	
	
	
	% Sección 1: Introducción
	\section{Introducción}
	

\begin{frame}
	\frametitle{1. Introducción}
	\begin{itemize}
		\item \textbf{Identificación de necesidades}:
		\begin{itemize}
			\item Usuario del ticket digital $\rightarrow$ no tiene informes de sus datos.
		\end{itemize}
		
		\item \textbf{Objetivos}:
		\begin{itemize}
			\item Proporcionar al usuario del ticket digital una herramienta que muestre en gráficos visuales:
			\begin{itemize}
				\item \textbf{Evolución de precios} (inflación) a lo largo del tiempo en los productos habitualmente obtenidos en el mismo establecimiento\footnote{La evolución de precios se mostrará solamente para un mismo centro de Mercadona, dado que distintos centros pueden cambiar los nombres de los productos (por ejemplo, en Cataluña…)}.
				\item \textbf{Evolución del gasto} total del usuario a lo largo del tiempo por períodos temporales.
			\end{itemize}
		\end{itemize}
	\end{itemize}
\end{frame}

			
			
			

			
			

	
	
	
	
	
		% Sección 2
		\section{Diseño}
	
	


			
			\subsection{Requisitos}
				% DIAPO REQUISITS
				\begin{frame}
					\frametitle{Requisitos de los usuarios}
					
					Que los usuarios tengan una cuenta de gmail con tickets digitales de Mercadona dentro e, idealmente, tenga decenas de tickets digitales: idealmente con compras estables y productos de adquisición recurrentes. El requisito indispensable es tener un mínimo de dos tickets digitales distintos.
		
				\end{frame}
				
		
			
	
			
			
		
				% DIAPO REQUISITS funcionals
				\begin{frame}
					\frametitle{Requisitos funcionales}
					
					
					\textbf{REQUISITO A:} Mostrar\textbf{ \textit{evolución de los precios}} de los productos unitarios adquiridos \underline{con más frecuencia} (visualizable en un gráfico donde en X tendremos el tiempo y en Y el precio en euros).
					
					\textbf{REQUISITO B:} Mostrar {\textbf{gasto total en distintas ventanas temporales}} del usuario: períodos de 1, 3, 6 meses y un año; independientemente del centro de Mercadona en el que se compre (todos juntos).
					
					\textbf{REQUISITO C:} Al lado del gasto total anterior se incluirá un \textbf{\textit{diagrama de sectores}} desglosando \underline{porcentaje de dinero} gastado en 13 categorias \href{https://shorturl.at/whzPf}{\color{blue}{(click para ver categorías)}}
					
				\end{frame}
				
				
				
							
				% DIAPO REQUISITS funcionals continuaco
				\begin{frame}
					\frametitle{Requisitos funcionales (cont.)}
					
				
					
					%REQUISIT D I E NO EREN A LA REDACCCIO DEL PROJECTE INICIAL CREC
					\textbf{REQUISITO D\footnote{Requisito añadido después de la presentación inicial del proyecto.}:} Los PDFs descargados del correo del usuario se almacenarán en una carpeta local del ordenador del usuario.
					
					\textbf{REQUISITO E\footnote{Requisito añadido después de la presentación del proyecto.}:} El sistema front-end y back-end de registro permitirá redirigir a los usuarios rápidamente a un registro de forma inteligente. Nos inspiraremos en el sistema de  registro e iniciar sesión de NetFlix. Ver diagrama enrutamiento.
	
				\end{frame}
			
			
			
									
						% DIAPO REQUISITS funcionals RESUM
				\begin{frame}
					\frametitle{Requisitos funcionales (RESUMEN)}
					
					
					De los requisitos al diseño (anticipo de lo que será el dashboard):
				
					\begin{itemize}
						\item evolución de precios por producto $\rightarrow$ \textbf{``inflalyzer''}
						\item gastos por categoría de alimentación $\rightarrow$  \textbf{``categoryzer''}
						\item ventanas temporales de gastos $\rightarrow$  \textbf{``intervalizer''}
					\end{itemize}
					
				\end{frame}
			
			
			
			
			
		
		
		
		
		
		
		
		
		\subsection{Diagramas de sistemas}
		
			% DIAPO DIAGRAMA DE SISTEMES GENERAL
			\begin{frame}
				\frametitle{Diagrama general}
				
				
				\begin{figure}
					\centering
					\includegraphics[width=1\linewidth]{../img/diagramaSistemesAplicacioMercapp}
					
					\label{fig:diagramasistemesaplicaciomercapp}
				\end{figure}
				
			\end{frame}
			
			
			% DIAPO DIAGRAMA SISTEMES: REGISTRE (SIMPLIFICAT)
			\begin{frame}
				\frametitle{registro}
				
				\begin{figure}
					\centering
					\includegraphics[width=1\linewidth]{../img/diagramaSistemesAplicacioMercappCAMIREGISTREbo}
					
					\label{fig:diagramasistemesaplicaciomercappcamiregistrebo}
				\end{figure}
				
			\end{frame}
			
			
			% DIAPO DIAGRAMA SISTEMES: REGISTRE (INICI SESSIÓ)
			\begin{frame}
				\frametitle{inicio de sesión}
				
				\begin{figure}
					\centering
					\includegraphics[width=1\linewidth]{../img/diagramaSistemesAplicacioMercappCAMIINICISESSIO}
					
					\label{fig:diagramasistemesaplicaciomercappcamiinicisessio}
				\end{figure}
				
			\end{frame}
			
		
		
		
		
	% Sección 3
	\section{Desarrollo}
	
	
	
	
	% FER AQUI LA DIAPO INICIAL PER MOSTRAR LES PARTS DE DESARROLLO
	
		
	%LA DIAPO QUE EXPLICA EL QUE HI HAAL DISENY

	
	
	
	\subsection{Entornos de desarrollo}
	\begin{frame}
		\frametitle{Entornos de desarrollo}
					
		\begin{table}[h!]
			\centering
			\begin{tabular}{|l|l|}
				\hline
				\textbf{Editor / Herramienta} & \textbf{Puerto\footnote{El host es localhost})} \\
				\hline
				IntelliJ IDEA (Java, SpringBoot) & 8080 \\
				VSCode (HTML, CSS, JS con Live Server) & 5500 \\
				VSCode (Python, con FastAPI\footnote{No depende del editor de código}) & 8000 \\
				MySQL Workbench & 3306 \\
				MongoDB Compass & 27017 \\
				\hline
			\end{tabular}
			\caption{Entornos de desarrollo y puertos utilizados para despliegue en local}
		\end{table}
	\end{frame}
	
	
	
	

	\subsection{Despliegue}
	\begin{frame}
		\frametitle{Despliegue}
		Se ha automatizado la creación de imágenes e instanciado de contenedores para cada microservicio. PUERTOS: ¡idem!
		
		\begin{figure}
			\centering
			\includegraphics[width=.7\linewidth]{../img/dockeritzacioAplicacioPlantilla}
			\label{fig:dockeritzacioaplicacioplantilla}
		\end{figure}
	\end{frame}
	
	
	\begin{frame}
		\frametitle{Despliegue (cont.)}
		
		
		\begin{table}[h!]
			\centering
			\begin{tabular}{|l|l|}
				\hline
				\textbf{Imagen original} & \textbf{puerto} \\
				\hline
				
				openjdk:17-alpine & 8080 \\
				nginx:alpine & 5500\footnote{localhost no sirve; usar 127.0.0.1 en navegador para ver index.html} \\
				Python:alpine (\href{https://shorturl.at/YdNuy}{\color{blue}{DF}}) & 8000 \\
		
				\hline
			\end{tabular}
			\caption{Imágenes docker base y puertos donde instanciamos su contenedor}
		\end{table}		
	
	
		\begin{table}[h!]
		\centering
			\begin{tabular}{|l|l|}
				\hline
				\textbf{base de datos} & \textbf{puerto} \\
				\hline
				
				\color{red}{MySQL} & 3306 \\
				\color{red}{MongoDB} & 27017 \\
		
				\hline
			\end{tabular}
			\caption{Bases de datos: no contenerizadas}
		\end{table}		
	\end{frame}
	
	
	
	
	\begin{frame}
		\frametitle{CONTINUAR PER 3.4 DE LA MEMORIA EN APARTAT DESARROLLO}
		\textbf{ometre dockeritzacio que surti a desarrollo de la memoria perque ja s'ha posat lo basic a disseny per no repetir.}
		\textbf{Posar sobretot estructures projectes i NO oblidar el diagrama d'enrutament.}
	\end{frame}
	
	
	
	
	\subsection{Spring Boot: gestión usuarios}
	
	
	
	\begin{frame}
		
		
		\begin{figure}
			\centering
			\includegraphics[width=1\linewidth]{imgEspecifiques/springBootLogoPART}
			
			\label{fig:springbootlogopart}
		\end{figure}
		
	\end{frame}
	
	
	\begin{frame}
		\begin{figure}
			\centering
			\includegraphics[width=1\linewidth]{imgEspecifiques/diaposolicitudTokenAccesFastAPI.png}
			\label{fig:diaposolicitudTokenAccesFastAPI}
			\caption{Función de servicio en Spring Boot para expedir un acceso con el id de usuario proporcionado por FastAPI (solicitud cliente cuando FastAPI ya tiene tickets persistidos). \textbf{Nótese que en 3 líneas de código tenemos la persistencia en mySQL hecha. ¿Función save?}.}
		\end{figure}
	\end{frame}
		
	
		\begin{frame}
		\begin{figure}
			\centering
			\includegraphics[width=1\linewidth]{imgEspecifiques/JPAexplicacioSaveRepoUsuari.png}
			\label{fig:JPAexplicacioSaveRepoUsuari}
			\caption{Con JpaRepository tenemos funciones para persistir en base de datos sin tener que definir consultas (la función save de la diapositiva anterior). Al hacer nuestra interface para operaciones de persistencia solamente debememos extender de JpaRepository e indicar la clase con la que hacemos el ORM y su clave primaria}
		\end{figure}
	\end{frame}
	
	
	
	
	\subsection{FastAPI: parseo de tickets}
	
	


		\begin{frame}
				
			\begin{figure}
				\centering
				\includegraphics[width=1\linewidth]{imgEspecifiques/fastAPIiPythonLogos}
				\label{fig:fastapiipythonlogos}
			\end{figure}
			
		\end{frame}


		\begin{frame}
			\begin{figure}
				\centering
				\includegraphics[width=1\linewidth]{imgEspecifiques/ticketExtraccioA.png}
				\label{fig:ticketExtraccioA}
			\end{figure}
		\end{frame}
		
	
		
		\begin{frame}
			\begin{figure}
				\centering
				\includegraphics[width=1\linewidth]{imgEspecifiques/ticketExtraccioB.png}
				\label{fig:ticketExtraccioB}
			\end{figure}
		\end{frame}
		
		\begin{frame}
			\begin{figure}
				\centering
				\includegraphics[width=1\linewidth]{imgEspecifiques/ticketExtraccioC.png}
				\label{fig:ticketExtraccioC}
			\end{figure}
		\end{frame}
		
		\begin{frame}
			\begin{figure}
				\centering
				\includegraphics[width=1\linewidth]{imgEspecifiques/ticketExtraccioD.png}
				\label{fig:ticketExtraccioD}
			\end{figure}
		\end{frame}
		
		\begin{frame}
			\begin{figure}
				\centering
				\includegraphics[width=1\linewidth]{imgEspecifiques/ticketExtraccioE.png}
				\label{fig:ticketExtraccioE}
			\end{figure}
		\end{frame}
		
		\begin{frame}
			\begin{figure}
				\centering
				\includegraphics[width=1\linewidth]{imgEspecifiques/ticketExtraccioF.png}
				\label{fig:ticketExtraccioF}
			\end{figure}
		\end{frame}
		
		\begin{frame}
			\begin{figure}
				\centering
				\includegraphics[width=1\linewidth]{imgEspecifiques/ticketExtraccioG.png}
				\label{fig:ticketExtraccioG}
			\end{figure}
		\end{frame}
		
	
				
		\begin{frame}
			\begin{figure}
				\centering
				\includegraphics[width=1\linewidth]{imgEspecifiques/ticketExtraccioH.png}
				\label{fig:ticketExtraccioH}
			\end{figure}
		\end{frame}
		
		
		\begin{frame}
			\begin{figure}
				\centering
				\includegraphics[width=1\linewidth]{imgEspecifiques/ticketExtraccioI.png}
				\label{fig:ticketExtraccioI}
			\end{figure}
		\end{frame}
		
		\begin{frame}
			\begin{figure}
				\centering
				\includegraphics[width=1\linewidth]{imgEspecifiques/ticketExtraccioJ.png}
				\label{fig:ticketExtraccioJ}
			\end{figure}
		\end{frame}
		
		\begin{frame}
			\begin{figure}
				\centering
				\includegraphics[width=1\linewidth]{imgEspecifiques/ticketExtraccioK.png}
				\label{fig:ticketExtraccioK}
			\end{figure}
		\end{frame}
			
	
			
		\begin{frame}
			\begin{figure}
				\centering
				\includegraphics[width=.8\linewidth]{imgEspecifiques/ticketExtraccioL.png}
				\label{fig:ticketExtraccioL}
			\end{figure}
		\end{frame}
		
		\begin{frame}
			\begin{figure}
				\centering
				\includegraphics[width=.8\linewidth]{imgEspecifiques/ticketExtraccioM.png}
				\label{fig:ticketExtraccioM}
			\end{figure}
		\end{frame}
	
		% CONTINUAR FAST API AQUI
	
	
	\begin{frame}	
		\frametitle{Delimitamos la tabla de productos (gracias a la cabecera)}
		\begin{figure}
			\centering
			\includegraphics[width=1\linewidth]{imgEspecifiques/ticketExtraccioN0}
			\caption{Este proceso depende de encontrar la cabecera en la línea siete (funciona para catalán y castellano). Lanzamos excepcion si falla.}
			\label{fig:ticketextraccionN0}
		\end{figure}
	\end{frame}
	
	
	
	
	
	
		\begin{frame}	
			\frametitle{1a Detección productos envasados (No granel)}
			\begin{figure}
				\centering
				\includegraphics[width=1\linewidth]{imgEspecifiques/ticketExtraccioN1}
				\caption{Procedimiento: primera aproximación a la detección de productos que no son a granel mediante su importe.}
				\label{fig:ticketextraccionN1}
			\end{figure}
		\end{frame}
		

	
		\begin{frame}	
			\begin{figure}
				\frametitle{1a Detección productos a granel}
				\centering
				\includegraphics[width=1\linewidth]{imgEspecifiques/ticketExtraccioN2}
				\caption{Procedimiento: primera aproximación a la detección de productos a granel a partir de la falta de importe en su primera línea.}
				\label{fig:ticketextraccionN2}
			\end{figure}
		\end{frame}
	
		\begin{frame}
			\frametitle{1a Detección productos envasados (continuación)}
			
			
			$\forall$ producto envasado $\exists$ ``d,dd'' \underline{al final de línea} (importe).
			
			\begin{itemize}
				\item Si se compra $\rightarrow$ \textbf{una unidad}, \textit{entonces}:
				\begin{itemize}
					\item \textbf{No} Existe patrón ``d,dd'' a su izquierda (columna P.Unit)?
				\end{itemize}
				\item Si se compran $\rightarrow$ \textbf{2 o más} unidades, \textit{entonces}:
				\begin{itemize}
					\item \textbf{Sí} Existe patrón ``d,dd'' a su izquierda (columna P.Unit)?
				\end{itemize}
			\end{itemize}
		\end{frame}
	
	
		\begin{frame}	
			\begin{figure}
				\centering
				\includegraphics[width=1\linewidth]{imgEspecifiques/ticketExtraccioN}
				\caption{\textcolor{red}{producto conflictivo envasado}: número de unidades queda mezclado con el inicio de la descripción o nombre de un producto imposibilitando segmentar ambos datos mediante espacio (\textit{split()})}
				\label{fig:ticketextraccionN}
			\end{figure}
		\end{frame}
		
		
		\begin{frame}	
			\begin{figure}
				\centering
				\includegraphics[width=1\linewidth]{imgEspecifiques/ticketExtraccioO}
				\caption{\textcolor{green}{Solución al conflicto}: se calcula qué parte de los dígitos pertenecen al número de unidades adquiridas y qué parte al nombre o descripción del mismo mediante coociente Importe/precioUnitario}
				\label{fig:ticketextraccionO}
			\end{figure}
		\end{frame}
			
			
	
	\begin{frame}	
		\begin{figure}
			\centering
			\includegraphics[width=.75\linewidth]{imgEspecifiques/ticketExtraccioP}
			\caption{\textcolor{red}{producto conflictivo}: Sale un parking que podemos confundir por un producto envasado (primera línea) y uno a granel (2a línea) que no tendría la línea que lo suele seguir con los datos a extraer.}
			\label{fig:ticketextraccionP}
		\end{figure}
	\end{frame}
	
	
	\begin{frame}	
		\begin{figure}
			\centering
			\includegraphics[width=.75\linewidth]{imgEspecifiques/ticketExtraccioQ}
			\caption{\textcolor{green}{Solución al conflicto}: Saltamos la línea que contiene "PARKING" y la siguiente sin llegar a procesar nada de su contenido: no es de interés.}
			\label{fig:ticketextraccionQ}
		\end{figure}
	\end{frame}
	
	
	\begin{frame}
		\begin{figure}
			\centering
			\includegraphics[width=1\linewidth]{imgEspecifiques/ticketExtraccioR}
			\caption{\textcolor{red}{producto conflictivo a granel}: El producto ocupa tres líneas en vez de dos. El conflicto viene por partida doble: se añade una línea por encima con la categoria y esta primera línea -y la segunda- NO tiene un número ``1'' de unidades como en el resto de productos a granel. }
			\label{fig:ticketextraccioR}
		\end{figure}
	\end{frame}
	
	
	\begin{frame}
		\begin{figure}
			\centering
			\includegraphics[width=1\linewidth]{imgEspecifiques/ticketExtraccioS}
			\caption{\textcolor{green}{Solución al conflicto}: \textcolor{orange}{No la vamos a implementar por ahora}, porque queremos forzar que salgan errores, tal y como sería para la aplicación en producción (tickets no vistos previamente, casos imposibles de preveer sin un enfoque empírico)}
			\label{fig:ticketextraccioS}
		\end{figure}
	\end{frame}
	
	\begin{frame}
		
		\begin{figure}
			\centering
			\includegraphics[width=1\linewidth]{imgEspecifiques/mostraErrorsPas4_i_maneig}
			\caption{Así mostraremos los tickets cuyo parseo ha fallado al usuario y los guardaremos a parte concatenando el id de usuario y el ticket}
			\label{fig:mostraerrorspas4imaneig}
		\end{figure}
		
	\end{frame}
	
	
	\begin{frame}

		\begin{figure}
			\centering
			\includegraphics[width=1\linewidth]{imgEspecifiques/ticketFinal}
			\caption{Ticket parseado correctamente a su formato JSON (persistible)}
			\label{fig:ticketFinal}
		\end{figure}
		
	\end{frame}
	
	
	
		\subsection{Front-end: Vanilla JS}
		
		\begin{frame}
			\frametitle{POSAR FRONTEND AQUI}
			
			
		\end{frame}
	
	
	
	
	
	
	
	
	
	
	
	
	
	
	
	
	
	
	
	
	
	
	
	
	
	
	
	% Sección 4
	\section{Conclusiones}
	
	\begin{frame}
		\frametitle{Conclusiones}
		\begin{itemize}
			\item Se ha aprendido a manejar tokens JWT
			\item etc etc
		\end{itemize}
	\end{frame}
	
	% Agradecimientos
	\begin{frame}
		\frametitle{Gracias por vuestra atención}
		¿Preguntas?
	\end{frame}
	
	
	
	
	
	
	
	
	

\end{document}
